%\documentstyle[12pt]{article}
\documentclass[12pt]{article}
\textwidth 165mm
\textheight 225mm
\oddsidemargin  1mm
\evensidemargin  1mm
\topmargin 1mm
%\usepackage[latin1]{inputenc}
%\pagestyle{empty}
\usepackage[firstpage]{draftwatermark}
\SetWatermarkScale{4}
\begin{document}

\begin{center}

\vspace{30mm}

{\Large \bf Definition of a Format for

Transfer, Update and Maintenance of

Phase Dependent Databases (PDB) in

Materials Science

}

\vspace{20mm}

SGTE\\
Bo Sundman\\
bo.sundman@gmail.com

\vspace{120mm}

Last update \today
\end{center}

\newpage

This document is based on a previous database format known as TDB
proposed by SGTE in 1990 for the interchange of thermodynamic model
parameters between different software.

This document defines a format for interchange of thermodynamic model
parameters and related data between different software and some advice
about structuring, updating and maintenance of a database using this
format.

\vspace{5mm}
{\bf Revision history}

\begin{tabular}{lll}
Date       & Revision    &  Notes \\
2016-09-16 & First draft & Bo Sundman\\
\today   & Second Draft & Bo Sundman's interpretation of some comments.
\end{tabular}

\vspace{5mm}

{\bf How to read this draft}

\bigskip

Please be aware that many definitions may require a more strict
formulation to avoid misinterpretations.  Please note of all of these
and propose a more strict formulation.

Please also suggest a better order of presentation.

The document is long but the first part contains many explanations
that is not necessary in order to use the database format.  All
essential user information can be found in the appendices.  Read those
first and go to the main document only when you do not understand or
agree.

\begin{itemize}
\item The general concept of thermodynamic databases is first
  described as this document may be of interest to other people than
  those currently dealing with thermodynamic databases.

\item Then some concepts needed for the format are described.

\item Then all keywords that is part of the database format are
  explained.  Many of these need additional information and this can
  be extended whenever there is a new model or parameter that should
  be included.  All such such information are documented in the
  Appendices to make it easier to modify.

\item Finally some advice for the database mangers and discussions of
  future developments.

\item The appendices describe all details for anyone who wants to use
  this database format.
\end{itemize}

\bigskip

{\bf Test software}

\bigskip

There is now a free thermodynamic software, OpenCalphad (OC), that can
be used for tests of a new format.  This software can read most
unencrypted TDB files and a subset of this software will be modified
to become a test software for the new format.

\vspace{10mm}

{\em I have inserted some explanations, comments and questions in
  italics in the text.  I propose to call this format PDB, Phase
  oriented DataBase}

\newpage

{\Large \bf Proposed changes after 2016.10.10}
\begin{itemize}
\item Bosse: Moved the description of the $T$ and $P$ dependence to an
  appendix.
\item Bosse: Added the D suffix after the table in Appendix F.
\item Bosse: Define errors messages and warnings.  There should be an
  error if the same phase occures more than once (not in include
  files) and also if a parameter is duplicated (what about include
  files?).  Constituents that are not entered as species is an error.
\end{itemize}

\newpage

\tableofcontents

\newpage

\section{General}

A thermodynamic database for materials science which deals with many
different kinds of solid solutions in multicomponent systems contains
data for elements and phases which make it possible to calculate the
equilibrium state of such a system.  Such a database can be used in
various software systems and in order to make it easy to interchange
data between different software systems this document defines a format
using a normal text file with a small number of keywords that can be
used to describe the content of the database.  In addition to the
thermodynamic data other information related to the phase and its
composition at various values of $T$ and $P$ can be stored using the
same basic format.

Each software which will use a database in this format should have an
interface which can read the database written in this format and store
the data in the internal format used by the software system.  It
should also have an interface to transform the internal data format of
the software system to a file using this format in order to make it
available to another software.

The format defined here can easily handle future extensions of the
types of data stored in the database, for example thermo-physical data,
mobilities etc. For more information on this see
sections~\ref{sec:paramid} and \ref{sec:material}.

\subsection{Phase or Composition Oriented Data Storage}

In thermodynamic databases for substances it is common to store data
according to the composition of a phase.  This means that the data for
pure Fe with bcc structure (ferrite), with fcc structure (austenite)
and liquid are all stored as one dataset for pure Fe.  For the given
composition a single expression is given for temperatures from 298.15
K and up.  This method is not suitable for solution databases for two
reasons:
\begin{enumerate}
\item The first reason is that for solutions it is necessary to
  extrapolate the thermodynamic data outside the stable range of the
  constituents because when other elements dissolve in the austenite,
  ferrite or liquid the stability range in $T$ changes for the phase.
  In a substance databases such extrapolations are only made for the
  gas but in a solution database there must be thermodynamic data for
  each condensed phase for the whole temperature range.  This means
  also that it is impossible to know the stable phase for a pure
  element at a given $T$ without calculate and compare the Gibbs
  energy of different phases for the pure element.

\item The second reason is that interaction parameters are needed to
  describe the non-ideal solution phases.  Such parameters
  are associated with two or more constituents but only one phase.
  Thus duplication and ambiguity is avoided if one stores these
  referred primarily to the phase.
\end{enumerate}

In solution databases one must in addition decide on a model to
describe the interaction between the constituents of a phase.  Many
different models are discussed in detail in \cite{07Luk} and here it
is sufficient to note that each phase can have a separate model.
However, a specific phase must be described with the same model
independent of its constituents because otherwise it would not be
possible to combine components in an arbitrary way.  This makes the
decision on the model to use for each phase of crucial importance in a
solution database.

\subsection{Composition dependence}\label{sec:tpdep}

In all solution models of the Gibbs energy the composition dependence
is expressed explicitly using a series expansion in the fractions of
the constituents of the phase.  Some phases can have internal degrees
of freedoms, for example a gas phase with many constituents containing
the same element, the equilibrium fraction of the constituents must be
determined by a minimization of the Gibbs energy.  In other cases a
phase may have contributions to the Gibbs energy from additions, the
value of which depend on other composition dependent properties.  One
example of this is the Inden model, described for example in
Lukas\cite{07Luk}, for the magnetic contribution to the Gibbs energy
due to ferromagnetic ordering.  In this case the composition
dependence of the critical temperature for magnetic ordering, and the
Bohr magneton number, must be described separately and for a given
composition the value of this critical temperature and the Bohr
magneton number are needed in order to calculate the magnetic
contribution to the Gibbs energy.

The interchange format described here can express the composition
dependence of any number of composition dependent quantities, in
addition to the Gibbs energy.  For each phase the same series
expansion in the fraction of the constituents of the phase is used for
all quantities.  The constituents of a phase can be the components of
the phase, molecules in a gas phase or constituents on different
sublattice sites in a crystalline phase.  The basic expression for
this series expansion is:
\begin{equation}
Z_M = \sum_i x_i ~^0Z_i + ~^EZ_M \label{eq:bascd}
\end{equation}
where $Z$ is the quantity which is composition dependent, $x_i$ is the
mole fraction of constituent i, $^oZ_i$ is the value of this quantity
for the pure constituent i (this value may be a function of
temperature and pressure).  The summation in eq.  \ref{eq:bascd}
represents a linear combination of the values for the constituents.
$~^EZ_M$ is the excess part which depends upon the interaction between
the constituents.  The subscript M means that the data are for one
mole of formula unit of the phase.

Note that eq.  \ref{eq:bascd} does not contain a term for the ideal
entropy of mixing, $RT\sum_i x_i\ln(x_i)$, because such a term is only
present in the Gibbs energy expression.  The ideal entropy does not
contain any adjustable parameters and it is not included in the
interchange format.

If a phase has sublattices there will be one additional summation in
the first term of eq.  \ref{eq:bascd} for each sublattice.  On the
sublattices one may have the same constituents (in order to describe
chemical ordering) or different (for example interstitial solutions or
inter-metallic compounds):
\begin{equation}
Z_M = \sum_i y_{i,1} \sum_j y_{j,2} ...  \sum_n y_{n,s} ~^0Z_{ij...n} + ~^EZ_M \label{eq:bassubl}
\end{equation}
where $y_{n,s}$ is the site fraction of constituent $n$ on sublattice
$s$.  For the sublattice model the configurational entropy expression
is given by:
\begin{equation}
R\sum_s a_s \sum_i y_{is} \ln(y_{is})
\end{equation}
where $a_s$ is the relative number of sites on sublattice $s$.   The
values of $a_s$ are given in the definition of the phase.

For a substitutional model the excess part of $Z_M$ must be zero
whenever the mole fraction of any constituent is unity.  The simplest
expression with this property is the regular solution model:
\begin{equation}
^EZ_M = \sum_i \sum_{j>i} x_i x_j Z_{ij} \label{eq:excess}
\end{equation}

The value of $Z_{ij}$ is a binary parameter for the system i-j.  One
may make $Z_{ij}$ temperature, pressure and composition dependent as
defined by the model index.  If there are sublattices it should be
noted that $i$ and $j$ must belong to the same sublattice and there
will be one additional summation in eq.  \ref{eq:excess} for each
sublattice.  The composition dependence of $Z_{ij}$ is often expressed
by a Redlich-Kister polynomial:
\begin{equation}
Z_{ij} = \sum_{\nu} (x_i - x_j)^{\nu} Z_{ij}^{\nu} \label{eq:rk}
\end{equation}
and one additional subscript, $\nu$, is needed to specify the place of
the parameter $Z_{ij}^{\nu}$ in the Redlich-Kister expression.  This
subscript is called the degree of the parameter.  There can be other
excess models and for these the degree can mean other things.  There
are no additional complications if one has sublattices also but note
that one may have composition dependent excess terms on each
sublattice separately.

One may have ternary parameters with three subscripts for constituents
on the same sublattice and in some cases even higher order excess
terms.  However, from eqs.  \ref{eq:bascd} to \ref{eq:rk} it should be
evident that the interchange format need only contain the parameters
$Z_{ij...}$, where the three dots mean the degree or additional
constituent indices, for each phase together with the model index and
the structure of the phase in order to reconstruct the full expression
for $Z_M$.  The place of a parameter in this expression is determined
by the phase, the constituents and the degree of the parameter given
as subscripts.

\subsection{Temperature and pressure dependence}

Temperature, $T$, and pressure, $P$, are potentials and most
parameters $Z_{ij...}$ in eqs.  \ref{eq:bascd} to \ref{eq:rk} are
functions of temperature and pressure.  For $T$-dependence the
function is often described by a power series but special models for
the low $T$ dependence can be used.  As many model parameters describe
metastable states it is important to restrict the $T$ dependence to
avoid unphysical features like a solid phase reappearing at very high
$T$.

Modeling the pressure dependence usually require a special model with
several parameters that may depend on $T, P$ and composition, for
example the thermal expansion and the bulk modulus.

\subsection{Helmholtz energy models}

This format is intended mainly for models that have explicit $P$
dependence and are thus not well suited for storing models using the
Helmholtz energy with explicit volume dependence.  The main reason is
that all use of the current TDB format is for Gibbs energy models.  It
would also be a complication to use a Helmholtz energy model because
$V$ is an extensive variable and thus depending on the size of the
system whereas $P$ is a potential.  But it may be possible to handle
such model in the future.

\section{Text Oriented Format}

This document use a simple text format for the interchange of the
thermodynamic data for a system.  Inside this text format specific
keywords are used for the structured data.  The big advantage with
this is that the database is readable by a human and can be edited
using a normal text editor.  In fact the database will normally be
updated by editing the database as a text file by a database manager.
The format should be designed to to make life easier for him or her,
see section~\ref{sec:manager}

The intention is that a freely available software will be developed to
check the syntax of a database using the format described here.  It is
up to the developer of a specific software to write an output form
this program to conform with his or her software.

All lines (except blank lines) that are comments should have a dollar
sign, ``\$'' as first non-blank character.

\subsection{Formatting the text}

There are some basic rules to simplify the reading, editing by a human
and also parsing the database by a software.

\subsubsection{Case insensitivity}

When a database using the interchange format is read by a computer
program all lower case letters should be treated as upper case, thus
Fe, FE and fe are the same name.  To make it easier to read by a human
upper and lower case can be mixed when editing the database file.

\subsubsection{Line length}

Some software have fixed length for reading a text line from a file
and a maximum length of a line in the database should not exceed 128
characters.

\subsubsection{Super- and subscripts}

Super and subscripts are not allowed in this format.  Appropriate ways
to handle cases where such are used in plain texts will be explained
here.  A parameter like
\begin{eqnarray}
^{\circ}G^{\rm bcc}_{\rm Fe} \nonumber
\end{eqnarray}
will be written as G(BCC,FE;0) (In fact the ``;0'' is redundant).  A
Redlich-Kister interaction parameter
\begin{eqnarray}
^{2}L^{\rm bcc}_{\rm Cr.Fe} \nonumber
\end{eqnarray}
is written G(BCC,CR,FE;2), (or L(BCC,CR,FE;2)) here the value after
the semicolon is important, see section~\ref{sec:comparr}.  It is
allowed to use L or G for these interaction energy parameters.

To describe the mobility of an element in a phase, for example Fe in bcc,
one may use a notation
\begin{eqnarray}
MQ^{\rm bcc}_{\rm Fe} \nonumber
\end{eqnarray}
and to describe the composition dependence of this property in the
Cr-Fe system one will need parameters like $(MQ^{\rm bcc}_{\rm Fe})_{\rm Fe}$
for the self-diffusion of Fe in BCC-Fe and $(MQ^{\rm bcc}_{\rm Fe})_{\rm Cr}$
for the diffusion of Fe at infinite dilution
in BCC-Cr.  This ``double subscript'' is handled by incorporating the
diffusing element in the parameter identifier like MQ\&Fe(BCC,FE) and
MQ\&Fe(BCC,CR) which is compatible with the notation used in the
current TDB format.

This does not mean all software must use this notation but they can
transform this notation to their own.

\subsection{Terminology: elements, species and phases}

Thermodynamics is used by many different sciences and each has
developed its own terminology.  Even within materials science there
are different use of the same terms and often strong feelings when a
term is used differently from what is considered correct.  However,
the TDB format introduced more than 25 years ago by SGTE has been
widely accepted within the CALPHAD community and also for publications
so a new format should be based on this.

In the TDB format the ELEMENT is a basic part of a thermodynamic
database.  In addition to the real elements in the periodic chart one
may use fictitious elements.  The electron, denoted ``/-'' and the
vacancy, denoted ``Va'' are also treated as elements but a system with
electrons and vacancies have as additional constraints that each phase
must be electrically neutral and that the chemical potential of
vacancies must be zero at equilibrium.  The elements has a unique one
or two letter symbol.

The elements can form SPECIES with a fixed stoichiometric ratio of the
elements.  Electrons and vacancies can be part of a species.  This use
of species is contested by some chemists who consider that a species
cannot exist without a phase designation.  For a chemist a species
H$_2$O must either be solid, liquid or gas but in this document a
species H2O is just a stoichiometric formula.  This species can be a
CONSTITUENT of many phases.

All data are stored in connection with a PHASE.  A phase has a name, a
model and species as constituents and the same species can be a
constituent of many phases.  This whole document is devoted to
describe how a PARAMETER for the different properties of a phase can
be stored depending on its model and constituents, $T$ and $P$.

In addition to the elements and species the term component is commonly
used in thermodynamics but that term not used in the database format
because components are related to the way a system is controlled by
setting conditions on the amounts or chemical potentials.

As described below this document provides a possibility to invent
names for species that is not necessarily the same as the
stoichiometric formula because there can be a problem when species
names are similar or have some structure information as discussed in
section~\ref{sec:spnames}.  For solid phases there is no general
standard for names but some recommendations will be given in
section~\ref{sec:phnames}.

\subsection{Nomenclature}

The description of the format used in this paper is simple and all
efforts have been used to avoid ambiguities.  A string enclosed within
brackets ``$<'' ``>$'' describes the type of information that should
appear there but other words or symbols used in the description should
be used literally.  An exception to this is necessary in order to
denote that a piece of information may be repeated.  Such information
is enclosed by slashes and sometimes the number of repetitions will
precede the slash.  In this case neither brackets nor slashes appear
in the actual data string.

All names, symbols, identifiers, codes and keywords defined in this
document may consist of the letters A-Z, the numbers 0-9 and the
underscore character ``\_'' in any order {\bf but must start with a
  letter}.  

A hyphen ``-'' can be used instead of underscore and must be treated
as identical to an underscore.

In a few cases, explicitly described below, names may contain other
characters or have a special form.  To make life easier for a database
manager a name can be abbreviated if the abbreviation is unique but it
is recommended that full length is used.  As already stated lower case
letters are not distinguished from upper case.

Numeric values that should appear in certain places according to this
format but which have no value should be indicated by the word
``NONE'' rather than ``0'' (zero).  In some cases it is also possible
to use the word ``UNASSESSED'' in as explained in section
\ref{sec:parameter}.

\subsection{Abbreviations}

Normally abbreviations of names, symbols, keywords etc should be
avoided but as the databases will be edited by humans it is natural
that they sometimes use abbreviations when editing the database.  As
the intention is to develop a software to check if a database conforms
to this document document this software will report ambiguities and
extend any abbreviations to its full form when requested to write the
database on a file.

\subsection{The model parameters}

The data stored in a thermodynamic database describe the Gibbs energy
of each individual phase as a function of the temperature, pressure
and the constitution of the phase.  The reason for selecting the Gibbs
energy rather than any other quantity is discussed in~\cite{78Hil}.

Already in the TDB format some additional properties that were needed
to calculate the Gibbs energy could be stored, for example the
critical $T$ for magnetic ordering and the Bohr magneton number.  In
this document many other properties that are related to the phase can
be stored, for example mobilities, individual Bohr magneton numbers
and Debye $T$.  See section~\ref{sec:paramid}.

In the database the elements, species, phases with their constitution
create the fundamental structure.  All kinds of data are stored as
model parameters for each phase for a specified constitution as
functions of the potentials $T$ and $P$.

\subsection{The SER reference state}

The first Calphad databases 40 years ago usually had one phase with
zero for all Gibbs energy parameters for the pure elements and used
only lattice stability values for the other phases.  With the SGTE
unary database in 1991 most databases started to use the Stable
Reference State (SER), i.e. the stable state at 298.15~K and 1 bar as
reference state.  The change made it possible to use the Calphad
databases also to calculate heat capacites and heat balance
equilibria.

Although it is still possible to have a database with only lattice
stabilities in a database following to this document that is
discouraged and will not be discussed further in this documentation.

\subsection{Extracting subsets of data}

The interface software should be able to extract a subset from the
database by specifying a set of elements and obtain a complete subset
of all species, phases, functions and model parameters.  It can be
quite complicated to extract a subset of a large database using a text
editor.

\section{Syntax of the data interchange format}\label{sec:keyword}

With one exception the interchange format consists of sequences of the
following form:

\begin{verbatim}
<keyword> <body of information> !
\end{verbatim}

The keyword must be the first non-blank part of a line after an LF or
CR.  It may be preceded by spaces and tab characters only.  The
exclamation mark, ``!''  is used to terminate the information that
belongs to the keyword and there must not be any exclamation mark
inside the body of information part.

Any control character between the keyword and the exclamation mark
will be ignored except TAB, FF, CR and LF which will be replaced by a
single space.

{\em In the original TDB format all CR and LF were ignored in the body
  of information (not replaced by space) because the strange behavior
  of electronic transfer between computers around 1990, often using
  raw X25.  In many cases LF and CR were inserted in arbitrary places,
  or after each 80 character, in the text that was transfered.  Today
  we can expect that this will not happen}

A keyword should be unique with its first 6 characters and its first
character must be alphabetic.  Note that the keyword itself must be
followed by at least one space .  The keywords currently defined in
this interchange format will be described in detail below and are
summarized in Appendix~B.

The order of the keywords fairly free, except for a few cases
specified below.  With one exception, the keywords must not be nested,
i.e.  after a keyword there must not be any other keywords until after
the terminating exclamation mark, ``!''.  It is possible to add
comments after the ! up to the end of the line.

Only one keyword can appear on one line but the body of information
may span several lines.

\subsection{Reading sequence}

The main idea is that it should be possible to read the database file
sequentially once to obtain all information.  This means there are a
few rules of the order of keywords:
\begin{itemize}
\item The VERSION keyword must be the first keyword.
\item An ELEMENT must be defined before it is used in a species or as
constituent.
\item A SPECIES must be defined before it is used as a constituent.
\item The TABLE\_OF\_MODELS must appear before any PHASE keyword.
\item The CONSTITUENT keyword must follow directly after the PHASE
  keyword.
\item The TABLE\_OF\_IDENTIFIERS must appear before any PARAMETER
  keyword.
\item A PARAMETER keyword for a phase must be after the keyword for
  the phase.
\item The FUNCTION keyword can appear anywhere.
\item The BIBLIOGRAPHY, DATABASE\_INFORMATION, 
  ASSESSED\_SYSTEMS, can appear anywhere but only once.
\item The SPECIAL, INCLUDE\_FILE and DEFAULTS keyword can appear
  anywhere several times.
\item The CHECKSUM keyword should be last if used.
\end{itemize}

The FUNCTION keyword is a bit problematic as it is a tedious work for
the database manager to ensure that a function is defined before it is
used in another function or parameter.  Thus it is recommended that
the software interface accepts the use of an undefined FUNCTION when
reading a PARAMETER or another FUNCTION.  If there are any such
undefined functions when the database has been read, the database file
can be rewound and searched again for just FUNCTION keywords for
missing functions.  This may have to be done a few times.

Even if all functions are read it would be necessary to have them in
calling order to avoid errors when entering a function calling another
function that is not yet entered.  So rewinding and just reading
FUNCTION keywords for missing functions will ease the work of the
database manager.

Some database managers prefer to order all parameters by phase, some
prefer to order them by binaries, ternaries etc.  This is up to the
manager.

\subsection{Keyword ELEMENT}

Elements are those included in the periodic chart.  They are
referenced by their chemical symbols, i.e.  their names are one or two
letters long.  It is possible to have hypothetical element names e.g.
A or Z.  A few extensions to the periodic chart are necessary.  Thus
vacant sites on a sublattice are treated as a component and the name
VA has been adopted.  At equilibrium the chemical potential of
vacancies must be zero.  Electrons are denoted by the combination /- and
a positive charge can be denoted by /+ or /- -.  At equilibrium the net charge of
all stable phases must be zero.

Upper and lower case letters are not distinguished and thus CO, Co, co
and cO can all be used to denote cobalt.  Note that this does not mean
that one cannot use lower case for the second letter in a particular
implementation of the database.  The interface software can change
this when it uploads data from the interchange format.

All elements have a defined state at 298.15 K which will be called
Stable Element Reference state (SER).  The name of this state (e.g.  FCC
or H2\_GAS etc.), the atomic mass, the entropy at 298.15 K and the
difference in enthalpy between 298.15 and 0 K for the element in that
same state will be given at the element keyword.

\begin{verbatim}
ELEMENT <element name> <SER state> <atomic mass>
    <enthalpy at 298.15 K - enthalpy at 0 K>
    <entropy  at 298.15 K> !
\end{verbatim}

The element name can be one or two characters.  Examples:

\begin{verbatim}
element fe bcc_a2                55.847  4489.0  27.280 !
element am double_hcp           243.06   6407.0  55.396 !
element h  1/2_mole_of_h2_gas     1.0079 4234.0  65.285 !
\end{verbatim}

Note that the values of $H_{298}-H_{0}$ and $S_{298}$ are never used
for calculations above 298.15~K.

The current interest to expand the database to 0~K using Debye or
Einstein models does not require any changes in this keyword because
such data will be given relative to the phase.

\subsection{Keyword SPECIES}

Species are aggregates of elements with fixed stoichiometry.  The
elements are the simplest types of species and they do not have to be
given as species also.  The name of a species is usually its chemical
formula but must not exceed 24 characters.  The stoichiometric factor
1 can be excluded if there is no possibility of misinterpreting.  This
means CO1 or CO is cobalt while carbon-monoxide must be written C1O or
C1O1 because we does not distinguish between upper and lower case
letters and cannot use ``Co'' for Cobalt and CO for carbon monoxide.
If fact all single letter element symbols should be followed by a
stoichiometric number except if it is the last element.

Note that in chemistry a stoichiometric formula has almost always a
phase designation like g (for gas), l (for liquid) or s (for solid)
and many chemists do not like using a stoichiometric formula without
such a designation.  In this documentation the same species can be a
constituent of many phases and it is convenient to define the
stoichiometry of the species separate from the phase.  A species which
is present in a particular phase is called a constituent of the phase.
A phase with fixed composition can have a single species as
constituent.

In this document a species can have a name that is different from its
stoichiometric formula.  This is useful to identify isomers and
simplify the name of a species with a complex stoichiometry.

Parenthesis are not allowed either in the name or in the
stoichiometric formula.  A period ``.'' is not allowed in the name of
a species but can of course be used to specify a non-integer
stoichiometry.  In the name of a specied a hyphen can be used to
indicate negative charge and a ``+'' can be used to indicate a
positive charge, although the name od a species does not have to have
any indication that it is charged.

\begin{verbatim}
SPECIES <species name> /<element name> <stoichiometric factor>/ !
\end{verbatim}

There should be no space between the element name and stoichiometric
factor.  Examples

\begin{verbatim}
species h2o		h2o1 !
species c1o		c1o1 !
species mg2sio4		mg1o2si0.5 ! Can be confusing for the user
species s1o2		s1o2 !
species o-2		o1/-1 !
species fe+2		fe1/--2 !
species s1o4-2		s1o4/-2 !
species c2cl2h4_cis	c2cl2h4 !  These two species -
species c2cl2h4_trans	c2cl2h4 !        are isomers
species this_is_a_very_long_name fe1 !     Allowed but not recommended.
species feo		fe0.987O !
species ax              al2ca4h26o20 ! A very short name
species hole            va/+1 !
\end{verbatim}

The elements in the stoichiometric formula do not have to be in
alphabetical order.  The stoichiometric factors can be real numbers.
A final stoichiometric value of 1 can be ignored.  An ion must have
its charge specified in the stoichiometric formula, a divalent
positive charge can be written ``/--2'' or ``/+2''.  No parenthesis
are allowed in species names as well as for the stoichiometry.
Vacancies are also species but an electron by itself is not allowed as
species but can be entered as a charged vacancy.

\subsubsection{Ambiguous species names}\label{sec:spnames}

Many species names are abbreviations of another species.  Thus exact
match is required when species are used as constituents in model
parameters.  A very long stoichiometric formula one can be given a
short name.

\subsection{Keyword TABLE\_OF\_MODELS}

There are 3 levels of codes that can be provided for a phase, the
MODEL, any ADDITION and possibly some DETAIL.  At least a MODEL code
must be specified for each phase.

All codes used for a phase in the database either MODEL, ADDITION or
DETAIL, must be given in this table.  The table may contain codes
that are not used in the current database.

After this keyword each line must start with MODEL, ADDITION or DETAIL
followed by a unique code.  The code must be unique among all of the
three types.

After the code a free text can be written within single or double
quotes.  This text is just informative but may be used to display
information for a user of the database.  A complete list of the codes
and detailed explanation are given in the Appendix~C, D and E.

{\small
\begin{verbatim}
TABLE_OF_MODELS
model IDEAL  'for example ideal gas'
model SUBRKM 'Substitutional Redlich-Kister_Muggianu'
model SUBPK  'Substitutional polymer_Kohler model'
model I2SL   'ionic_liquid_two-sublattice model'
model KFGL   "Kaphor-Frohberg-Gaye-Lehmann irsid slag model"  
model CEFj   'Compound Energy Formalism with j sublattices'
model CEFjD  'CEF with j sublattices and disordered part'
model CEF4F  'CEF with 4 sublattices for FCC ordering and permutations'
model CEF4B  'CEF with 4 sublattices for BCC ordering and permutations'
model CEF5F  'CEF with 4 subl for FCC ordering, permutations and interstitials'
model CEF5BD 'CEF with 4 subl for BCC ordering, permutations, interstitials and disordered part'
model GUTS   'FactSage liquid model'
model BOND   'Model based on bond energies between sublattices'
addition IMAGB  'Inden magnetic ordering model for BCC'
addition IMAGF  'Inden magnetic ordering model for FCC and HCP'
addition IWMAGB 'Inden-Wei magnetic ordering model for BCC'
addition IWMAGF 'Inden-Wei magnetic ordering model for FCC and HCP'
addition LOWTD  'Low T Debye model'
addition LOWTE 'Low T Einstein model'
addition MURN   'Murnaghan pressure-volume model'
detail MGAP  'Miscibility gap with major constituents specified'
detail DCON  'Default major constituents specified in all sublattices'
detail STRUCT 'additional information about the structure of the phase'
!
\end{verbatim}
}

This is quite clumsy but normally one does not have to edit this
very often.  When the table is edited one can use new lines freely as
they are ignored when reading the body of information.

When a code is used after a phase additional information
can be provided within parenthesis as explained below.  The
expressions how to calculated the Gibbs energy for the various models
are given in Appendix~C and additions in Appendix~D.

\subsubsection{The model code}\label{sec:modcode}

This code need more information to be understood by the software.
Normally the software must have the basic model implemented when
reading the database, otherwise a warning or error message must be
given that the phase will be ignored.

There is no dilute solution model or model for the activity
coefficients but such models can be converted to a model using the
integral Gibbs energy using for example the paper by
Hillert\cite{86Hil} or Pelton and Bale\cite{86Pel}.

For all CEF models there must be a number 1 to 9 directly after CEF
specifying the number of sublattices (one may use 1 to have more than
one site in the substitutional lattice) and then within parenthesis
the number of sites on all sublattices.  For example

\begin{verbatim}
PHASE SIGMA CEF5(2.0 4.0 8.0 8.0 8.0) !
\end{verbatim}

The letters F and B after the number of sublattices denote the special
case when the software should take care of model parameter
permutations.  In a 4 sublattice ordered fcc phase a parameter like
$~^{\circ}G_{\rm Al:Ni:Ni:Ni}$ has 4 permutations that should have
identical values.  This is specified by the F.  The B means the same
for the ordered bcc phase (which has s different set of permutations).
The 4 sublattice hcp phase has the same permutations as fcc.

The letter D means that there is a separate set of model parameters
for the substitutional disordered state where the 4 ordered sublattices
has merged.  The model equation for this can be found in Appendix~D.
If there is an interstitial sublattice that is separate also in the
disordered phase.

\begin{verbatim}
PHASE BCC_4SL CEF5FD(0.25 0.25 0.25 0.25 3) IMAG!
\end{verbatim}

A software that does not handle permutations will have to expand these
automatically when reading the parameters for this model.

The parameters for the substitutional disordered part can be detected
by the number of sublattices in the constituent array of the
parameter.  But to be on the safe side this format requires that such
parameters have a suffix D on the parameter identifier.  See
section~\ref{sec:suffixd}.

For each model identifier a written documentation how to implement the
model must be given in Appendix~C of this document, preferably with a
reference to a published paper.

{\em There are many complicated models that may need extra
  information.  Unsymmetrical models like Toop or power series need
  additional information and I am not sure how to specify that.}


\subsubsection{The addition code}

Additions are contributions to the Gibbs energy due to some additional
physical model.  The IMAGB and IMAGF additions are well established.
The others are not yet implemented in most software.

There can be several additions for a phase.  For each addition
identifier a written documentation how to implement the model must be
given in this document.  The defined additions will be given in
Appendix~D.

\subsubsection{The detail code}\label{sec:detail}

The details given for a phase may be ignored by the software
reading the database but it can be useful in some cases.  One
detail can be additional structure information like

\begin{verbatim}
PHASE HALITE CEF2(1.0 1.0) IMAGF DETAIL(STRUCT(B2, PERICLASE)) !
\end{verbatim}

In some cases the details may be useful to generating start
constitutions as it is not always possible to calculate a global
equilibrium, for example if $T$ is not set as a condition.

For example the MGAP can be used to indicate there can be a
miscibility gap in the phase and provide default state values.  In the
example below it is indicated that the FCC phase can be stable with
mainly Ti or V in first sublattice and C and N in second.  There may
even be a proposal for a modified phase name (with a pre- or suffix),
MC\_CARB, for this phase included.

\begin{verbatim}
PHASE FCC_A2 CEF2(1.0 1.0) IMAGF MGAP(MC: TI V : C N:CARB) !
\end{verbatim}

Note that a phase may have a miscibility gap even if there is no MGAP
specified.

The DCON can also be handled by the \% suffix as described in
section~\ref{sec:const}.  These may be considered to be a way to help
the calculation software to find the equilibrium.  It can also be used
to avoid problems to identify which phase is the carbide and which is
the austenite!

There can be several details for a phase.  For each detail code a
written documentation is given in Appendix~E.

\subsection{Keyword PHASE}

Phases are homogeneous parts of the thermodynamic system with uniform
structure and composition.  In some cases one may use the word phase
also to denote parts which have uniform structure but non-uniform
composition although such a system cannot be in equilibrium.  For
solutions the phase is the central concept as data are stored
primarily according to the phase it is associated with.

\begin{verbatim}
PHASE <phase name> /<model>/ /<addition>/ /<detail>/ !
\end{verbatim}

The ``model'', ``addition'' and ``detail'' are explained below.  Some
of them have additional information.

Examples, note the text after ! are comments.

\begin{verbatim}
phase bcc_a2 SUBRKM IMAGB ! Substitutional solution and Inden magnetic bcc model
phase fcc_4SL CEF4F(0.25 0.25 0.25 0.25) IMAGF !  CEF with 4 ordered sublattices
                   and the Inden magnetic fcc model
phase gas IDEAL !  Ideal gas
phase wustite CEF2(1.0 1.0) IMAGF ! 
phase m23c6 CEF3(2.0 21.0 6.0) !  carbide with 3 sublattices
\end{verbatim}

In the previous TDB format the ``type defintion'' keyword was used for
many strange things and in this format we have tried to be more
explicit about the models and introduced model specifications that
should be explained in the database file.

The constituents of a phase are a subset of the species.  This subset
is given either by a ``constituent'' keyword.

{\em In the TDB format the CONSTITUENT keyword was optional as the
  constituents could be be deduced from the PARAMETER keywords.
  However, this is not allowed in this document as it is easy to make
  spelling mistakes or use wrong abbreviated species names.}

\subsubsection{Phase names}\label{sec:phnames}

There is no generally excepted way to name a phase.  For an ideal gas
the name ``gas'' is obvious and for a substitutional liquid one may
use ``liquid'' but there are several different models for liquids.
For solid stoichiometric phases one may use the chemical formula but
if there are several allotrophs it is not so convenient to have
suffixes like \_S1 and \_S2 for that.  For crystalline solution phases
one may use the lattice type like FCC or BCT but there are many
different FCC phases with various interstitial places filled with
other atoms or vacancies.  The Structur Bericht is useful but it is
far from complete.  For some phases there are established names like
Laves\_Phase or MU\_phase.  In general it is not convenient to use
transcriptions of greek letters as ALPHA which can mean different
phases in different system.  Even if exact match is required to identify
a phase one should avoid having a phase name which is an abbreviation
of another phase name.


\subsection{Keyword CONSTITUENT}\label{sec:const}

The constituent keyword must follow directly after the corresponding
PHASE keyword.  The syntax is

\begin{verbatim}
CONSTITUENT <phase name> : <species on first sublattice> 
/ : <species on other sublattices in sublattice order>/ : !
\end{verbatim}

This keyword must follow directly after the PHASE keyword but anyway
the phase must be specified.  The number of sublattices with
constituents must correspond to the specified for the phase keyword.
Note that colon ``:'' are used between the sublattices and comma ``,''
or a space between constituents in the same sublattice

Examples

\begin{verbatim}
phase gase ideal !
constituent gas :c c1h4 c2 c2h2 c2h2o1 c1o c1o2 h h2 h2o h2o2 o o2 o3: !
phase fcc subrkm !
constituent fcc :fe al si: !
phase wustite cef2(1.0 1.0) !
constituent wustite :fe+2 fe+3 va: o-2 : !
\end{verbatim}

\subsubsection{Range of stable phase composition}\label{sec:phaserange}

In a general thermodynamic database many phases will dissolve all
constituents.  When retrieving data from the database for a particular
subsystem there will be a large number of phases included that are
never stable because the set of stable phases This is due to the fact
that phases may dissolve many constituents but the phase may be stable
only for small additions of many of the constituents.  The set of
stable phases will of course change with $T, P$ and composition of the
system and this can only be determined by an equilibrium calculation.

For example in the Cu-Zn system neither Cu nor Zn exist as pure in the
bcc phase but nevertheless there are four bcc related phases stable in
the middle of this system.

In order to facilitate for the calculation software the CONSTITUENT
keyword can be used in order to specify MAJOR constituents on each
sublattice of a phase.  The major constituents should have a suffix
consisting of a per-cent sign ``\%''.  For example:

\begin{verbatim}
CONSTITUENT FCC_A1 :FE% CR NI% MN% S: VA% C: !
\end{verbatim}

Where Fe, Ni and Mn are major in the first sublattice and Va in the
second (interstitial) sublattice.  There is an alternative way to
specify the same information as described in section~\ref{sec:detail}.

There is no facility to specify limits of the composition range of a
phase.  This may be a feature tempting facility to implement but would
be very complicated to handle in a multicomponent database.  It is
very complicated for a software to handle if the composition of a
stable phase cross such a stability limit for a component.  Should the
calculation terminate with an error message?  The calculation may 
occur during a long time simulation where the phase is transforming.

It is the task of the database manager to verify that an assessed
subsystem does not create problems when added to a multicomponent
database.  Frequently additional model parameters have to be added or
even assessed parameters slightly modified.  See also
section~\ref{sec:info}.

\subsection{Keyword TABLE\_OF\_IDENTIFIERS}\label{sec:paramid}

A model parameter identifier determine which property the parameter is
associated with.  In the current TDB files these are normally those
listed in section~\ref{sec:tdbid} but in the new format this will be
considerably extended.

A parameter identifier describe a property that can be composition
dependent using the same model as the Gibbs energy.  Some of them can
also depend on $T, P$ and be specific for a particular element or
constituent.  

We should use short identifiers, at present the maximum length is 4
letters and we should try to keep that limit.  But it is important to
have unique identifiers!

A special keyword should list all the parameter identifiers used in a
database before the first parameter in the database.  All identifiers
that may be used are listed in Appendix~F.
\begin{verbatim}
TABLE_OF_IDENTIFIERS /<identifier> {<explanation>}/ !
\end{verbatim}

The table is useful to check for typing errors at least.  

It is important to establish a syntax for all identifiers as they may
depend also on the constituents of the phase even if several of the
identifiers may not have any associated code in a some software.  In
such a case the software can just ignore the parameters with this
identifier.

\subsection{Keyword PARAMETER}\label{sec:parameter}

All temperature and pressure dependent data for a phase is given using
the PARAMETER keyword.  With this keyword a phase and a composition or
composition range must be specified followed by an expression.  In the
simplest case the parameter gives the Gibbs energy of formation of a
phase with a single constituent.  In more sophisticated models the
parameter may be part of a complex Gibbs energy expression.  Note that
the Gibbs energy expression of a solution phase may have many
coefficients in its composition dependence.  Each of these
coefficients will appear in a PARAMETER keyword, giving the
temperature and pressure dependence of one coefficient.  See section
\ref{sec:tpdep} for more information.  The syntax is

\begin{verbatim}
PARAMETER <identifier> ( <phase name>, <component array> ; <degree> )
    <expression> <bibliographic reference> !
\end{verbatim}

The first part of the information after the keyword PARAMETER but
before the equality sign is the ``name'' of the parameter.  The
$<$identifier$>$ after the PARAMETER denote the type of composition
dependent quantity and it must be standardized.  The following
identifiers are used currently:

\subsubsection{The identifier}\label{sec:tdbid}

The identifier gives the type of parameter.  In the TDB files these
were limited to a very few:

\begin{tabular}{lll}
Symbol & Unit &      Notes\\\hline
G      & J/mol formula unit & Energy parameters that are part of the explicit\\
 && composition dependence of the Gibbs energy\\
L     & J/mol formula unit & Identical to G parameters but mainly used for\\
 && interaction parameters\\
TC    & Kelvin & Curie temperature parameters\\
BMAG & dimensionless & Bohr magneton parameters\\\hline
\end{tabular}

\bigskip

In the new format this will be extended considerably as described
in the section~\ref{sec:paramid} and Appendix~F.

\subsubsection{The component array}\label{sec:comparr}

The syntax of the $<$component array$>$ in the PARAMETER information is
composed of names of species separated by comma ``,''for species in the
same sublattice and using colon ``:'' to separate groups of species that
go into different sublattices.  A recursive definition is:

\begin{verbatim}
<component array> is
             <species name> or
             <species name>,<component array> when there are more species in
                   the same sublattice,
             <species name>:<component array> when there are more sublattices.
\end{verbatim}

There must not be any spaces in a component array.  Note that the
parameter in the Gibbs energy expression should be multiplied with the
fraction of the constituents given by the component array.

Whenever the value of a model parameter may depend on the order of the
constituents, the alphabetical order will be used unless something
else is explicitly stated, see the section~\ref{sec:defaults}.

The $<$degree$>$ in a parameter can be void or a value from 0 (zero)
to 9.  If the parameter is a binary interaction parameter in a
Redlich-Kister excess model the degree means the degree as a
coefficient in a Redlich-Kister polynomial.  For other parameters and
models the degree has different meanings.  In some cases the parameter
depend on the order of the constituents as discussed in Appendix~G.

\subsubsection{The sign of the sub-regular parameter}\label{sec:subreg}

In the Redlich-Kister polynom the sign of the parameter for odd powers
depend of the order of the interacting constituents.  By default in
this format the constituents will always be ordered alphabetically
according to the species names.

The database manager can always arrange the sign to be correct but if
he or she prefers that the sign should depend on the order the
constituents are written in the constituent array.

For example:\\ PARAMETER L(FCC,FE,CR;1) will by default be multiplied
with $(x_{\rm Cr}-x_{\rm Fe})$\\ If the database manager wants it to
be multiplied with $(x_{\rm Fe}-x_{\rm Cr})$\\ that must be specified in
the DEFAULTS keyword.  Such a change will apply to the whole database.

\subsubsection{Wildcard constituents}

It is possible to specify ``*'' in a sublattice to indicate that the
parameter is independent of the constitution on this sublattice.

Parameters with a wildcard in a sublattice will be added together with
parameters which also have a specific constituent in that sublattice.
Thus both the parameters

\begin{verbatim}
PARAMETER L(FCC,FE:*:VA) ...
PARAMETER L(FCC,FE:C,VA) ...
\end{verbatim}

should be included in a calculation of the Gibbs energy of the FCC
phase for a system with C and Fe.

\subsubsection{Parameters for a disordered part}\label{sec:suffixd}

As explained above in section~\ref{sec:modcode} there may be parameters
with for an ordered phase that belong to a disordered part.  These
will have a suffix D on the parameter identifier in addition to have
fewer sublattices.

\begin{verbatim}
PARAMETER G(FCC_4SL,AL:NI:NI:NI:VA) ...
PARAMETER GD(FCC_4SL,NI:C,VA) ....
\end{verbatim}
where the first parameter is part of the ordered description of the
FCC phase and the second is in the disordered part.  (In the TDB
format the disordered part is a separate phase that is connected by a
TYPE\_DEFINITION but that is a software dependent feature.)

\subsubsection{The expression}

The expression will contain coefficients which should be multiplied
with temperature and pressure.  The actual form of the expression is
described in Appendix~H, section~\ref{sec:exptype}.

Note that it may be important in some cases to distinguish between
parameters that are zero or those which has not been assessed.  A
parameter that is zero could normally be omitted completely but it is
recommended that unary and binary parameters should be included even
if they are zero.  The reason for this is that all parameters possible
according to the model that are not included in the database by
default will be considered as un-assessed.

If one particularly wants to stress that a parameter is not assessed
the function can have the value UNASSESSED.

\subsubsection{Bibliographic reference}

The $<$bibliographic reference$>$ should be found inside the body of
the bibliographic keyword and will normally refer to the paper where
this parameter was assessed or to the person who entered the
parameter, hopefully with an explanation.  See further in
section~\ref{sec:bib}.


\subsection{Keyword FUNCTION}

This keyword is
useful because many thermodynamic parameters are related.  For example
the Gibbs energy of formation of metastable phases is often based on
the Gibbs energy of the stable phase which can be entered
as a function and used in many parameters.

\begin{verbatim}
FUNCTION <name> <expression> !
\end{verbatim}

The name of a function can be used as symbol name in expressions.  In
the current TDB file the length is limited to 8 and it must start with
a letter.  In OC this is now extended to 16 characters and the
proposal is to allow 16 characters in the new PDB format.

NOTE that function names must never be abbreviated!

The expression is described in Appendix~H, section~\ref{sec:exptype}.
In the example below the function GHSERCR describes the Gibbs energy
of BCC-Cr at 1 bar from 298.15 to 6000~K.  The function GCRFCC
describes the Gibbs energy for metastable FCC-Cr using the same heat
capacity as for BCC-Cr, there is just a $\Delta H$ and $\Delta S$
added.  These functions are later used when describing the properties
of Cr in various phases.  Note that the PARAMETER keyword have a
bibliographic reference but not the FUNCTION.  There is no need for a
final hash caracter, ``\#'' after a function name.

\begin{verbatim}
FUNCTION GHSERCR    2.98150E+02  -8856.94+157.48*T-26.908*T*LN(T)
     +.00189435*T**2-1.47721E-06*T**3+139250*T**(-1);  2.18000E+03  Y
      -34869.344+344.18*T-50*T*LN(T)-2.88526E+32*T**(-9);  6.00000E+03  N !
FUNCTION GCRFCC     2.98150E+02  +7284+.163*T+GHSERCR;   6.00000E+03   N !

...

PARAMETER G(BCC_A2,CR:VA;0)  2.98150E+02  +GHSERCR;  6.00000E+03   N REF283 !

...

PARAMETER G(FCC_A1,CR:VA;0)  2.98150E+02  +GCRFCC;   6.00000E+03   N REF281 !
PARAMETER TC(FCC_A1,CR:VA;0)  2.98150E+02  -1109;   6.00000E+03 N   REF281 !
PARAMETER BMAG(FCC_A1,CR:VA;0)  2.98150E+02  -2.46;   6.00000E+03   N REF281 !

...

PARAMETER G(SIGMA,FE:CR:CR;0)  2.98150E+02  +8*GFEFCC+22*GHSERCR+92300
  -95.96*T;   6.00000E+03   N REF107 !

\end{verbatim}

It is illegal for a function to refer to itself in the expression
part.  Circular references are not allowed either.

In order to represent sqrt(T), i.e the square root of T, one must use
two functions

\begin{verbatim}
FUNCTION F1    1 0.5*LN(T) 6000 N; !
FUNCTION SQRTT 1 EXP(F1); 6000 N !
\end{verbatim}

Then the function SQRTT can be used wherever SQRT(T) is needed.  The
low $T$ limit must not be less than 1 as LOG of zero is -Infinity.

Some software may not use functions but it is always possible add up
the coefficients with the same $T$ dependence, possibly multiplied
with some coefficient, in order to obtain the expression for a
parameter.

\subsection{Keyword DATABASE\_INFORMATION}\label{sec:info}

After this keyword follows a free text (without any !) where the
database manager can inform a user about the database.  Typically it
should inform about the purpose of the database, the name of the
database manager(s), the elements in the database, last update, what
type of data it contains and preferably an estimate of the range of
validity.  See also section~\ref{sec:assys}.

\begin{verbatim}
DATABASE_INFORMATION <free text without any !> !
\end{verbatim}

There should be some way to format this text so it is not written as a
very long line.  Suggestions are welcome.

\subsection{Keyword BIBLIOGRAPHY}\label{sec:bib}

Most parameters in the database should originate from some assessment
and it is vital to give a reference to the paper or report where the
parameter was determined.  The reference must be given for each
parameter as it is not unusual that some parameters from a published
assessment are later modified to give a better extrapolation to a
higher order system.  If each parameter has a bibliographic reference
those changed have their reference modified at the same time.  It is
thus not sufficient to have a single bibliographic reference for a
binary system, it must be possible to have individual references for
all parameters.

Parameters that are modified manually should have as bibliographic
reference the date of the change, the name of the responsible and the
reason.  Please think of the database manager that will replace you,
without such references he or she will be completely lost.

It is important to keep the bibliographic information in the same file
as the parameters and the database manager never has time to update
two or more files simultaneously.

The syntax of the bibliographic keyword is

\begin{verbatim}
BIBLIOGRAPHY /<bibliographic reference> '<text>'/ !
\end{verbatim}

The $<$bibliographic reference$>$ is the same as used in the PARAMETER
keyword.  The text should give the publication or where to find the
report or simply '2016.09.10 Bo Sundman to make HCP less stable'.  The
text must be enclosed within single or double quotes because all
bibliographic references are defined in the same keyword.

A bibliographic reference may start with a number, for example
``91Din'' but one should use only letters and digits.  It should not
be case sensitive as nothing else is.

{\em Maybe it is better to enclose the text by curly braces {} as
  a missing ' (or added) may upset reading of the references.  But we
  use curly braces in references to papers?}

\subsection{Keyword SPECIAL}

This keyword is intended for parts of a database that include data
which is associated with a particular software and does not fit the
current format.  The syntax of this is a bit special as it may include
keyword sequences that should be ignored by other software.

Thus the SPECIAL keyword must be terminated by a double exclamation
mark, ``!!'', two ! following directly after each other.  It is thus
allowed to have single exclamation marks inside its body.

\begin{verbatim} 
SPECIAL <software id> <body of text not including double !> !!
\end{verbatim} 

Of course it would be rather meaningless if the whole database was
enclosed by a SPECIAL keyword.  All cases of SPECIAL should eventually
be replaced by accepted keywords.

\subsection{Keyword VERSION}

If this format will be used it will gradually include new features and
some may become obsolete or changed.  When reading a database it is
thus important to know for which version it was created.  At the same
time one can specify some default values for the high and low $T$
limits for the data.

\begin{verbatim}
VERSION <date> <default low T limit> <default high T limit> !
\end{verbatim}

\subsection{Keyword CHECKSUM}

{\em The checksum was never been implemented in TDB, I think it would
  be a good idea to have one if anyone can come up with a simple
  algorithm.  The checksum should be calculated for the part between
  keyword and ! after converting the text to upper case and removing
  the control characters as specified above.}

\subsection{Keyword DEFAULTS}\label{sec:defaults}

At present there is only one use for this keyword:\\

By default the odd parameters in a Redlich-Kister polynomial are
multiplied with the difference of the constituents in alphabetical order.
By specifying:
\begin{verbatim}
DEFAULT RKORDER=NOTALPHABETIC !
\end{verbatim}
after this change the order of the elements when calculating the
difference will be that specified in the constituent array.  This will
apply to all parameters read after this keyword.

The alphabetical order can be reset by:
\begin{verbatim}
DEFAULT RKORDER=ALPHABETIC !
\end{verbatim}

\subsection{Keyword ASSESSED\_SYSTEMS}\label{sec:assys}

The database manager can provide information about the assessed
systems in the database.  Possibly also some information how to
calculate them with his or her preferred software within ().  The
elements should be given in alphabetical order.

\begin{verbatim}
ASSESSED_SYSTEMS AL-FE AL-NI AL-SI FE-MN AL-MN AL-FE-MN !
\end{verbatim}

\subsection{Keyword INCLUDE\_FILE}

This keyword can be used anywhere and means that the file will be
opened and read as a database file.  Redundant keywords will be
ignored but new elements, species and phases will be added.  New
constituents for an already entered phase will be added provided the
models are the same.  New functions and parameters will be added.

{\em Question: should existing functions and parameters be overwritten
  or ignored or give error message?}

\section{Database manager tasks}\label{sec:manager}

A large solution database is created by adding individual assessments
of the pure elements, binary, ternary and higher order systems.
The assessment procedure is not part of this document, read for
example~\cite{07Luk}

\subsection{The beginning}

Normally a database starts from a few binary systems with the same
models for the phases and where extrapolations can be easily checked
but when it starts to grow by adding new elements and assessments the
task to verify the database becomes cumbersome.

\subsubsection{Unary data}

Until today the unary data has been very easy using the SGTE unary
database\cite{91Din}.  As there is considerable efforts today to model
the Gibbs energy down to 0~K and eliminate the unphysical breakpoints
at the melting $T$ of the elements this will maybe be more difficult
in the future.

\subsection{Adding an assessment}

When adding an an assessment the manager must unify the names of
phases and their models with those used in the database.

\subsubsection{Phase names}

When adding an assessment one should take care that phases that can
form a solution have the same name as well as model.  The current
databases use a mix of phase names which may seem strange to a
beginner but which is usually easy to understand with some experience.
Thus FCC\_A1 (where A1 is the structur bericht) means a phase with a
substitutional FCC lattice but which can have interstitials, like C or
N, in the octahedral sites, although this should actually have
structur bericht B1 which is represented for example by the TiC cubic
carbide.  But the same lattice with ionic constituents, like MgO, is
called HALITE as that is the name of the NaCl structure.  However, in
the C-O-Ti system there is a complete solubility between the TiC phase
called FCC\_A1 and the high temperature form of TiO which is called
HALITE.

There is also a phase CaF$_2$ where Ca occupies an FCC lattice and F
tetrahedral interstitial sites.  This has structurbericht C1 and
UO$_2$ has the same structure but there is no solubility between
CaF$_2$ and UO$_2$ and they should be given different names.

So the the field is open for anyone to argue about the best name for a
phase.  What is important is that phases in different systems that
have complete solubility with phases in other systems should be
treated as the same phase.

\subsubsection{Model compatibility}

Many inter-metallic phases have slightly different models in different
systems, for example the ordered B1, the $\mu$ phase and the $\sigma$
phases.  As it is important not to have several $\mu$ phases it can be
rather challenging to select model and modify the parameters in those
systems that have another model.

This will probably become more and more challenging as new assessments
using ordered descriptions of FCC and BCC phases are published.

\subsubsection{Simple checks}

This may require adding parameters for un-assessed binaries and for
phases which dissolve all elements in the new assessment but is not
stable inside the assessed system to avoid that this phase becomes
stable where it should not be, for example adding a positive
interaction to HCP in an assessment of the Fe-Cu system, where the HCP
is not stable.

For binary systems the Gibbs energy curves at various $T$ can reveal
strange things.

\subsubsection{More subtle checks}

The story of the interaction parameter for FCC in the Cr-Mo system.

A useful method to find strange parameters is to calculate binary or
ternary diagrams suspending all but one or a few phases.

Calculating phase diagrams at several temperatures extrapolating the
new assessed systems with the other elements in the database, one at a
time may reveal strange things.

\subsection{Estimated range of validity}

There is no way a large solution database can be maintained without a
skilled manager.  This manager should of course have many
computational tools as help but it is his or her skills that determine
how easy and accurate the database is when used by industry or for
academic research.

One task of the manager is to estimate limits for the composition for
which he or she can believe the database will give reasonable results.
Feedback from users is very important for this at is is impossible to
check things without experimental data.

\section{Some future features}

This section is intended for collection of ideas that is not part of
the formal definition but may guide future extensions.

\subsection{Databases generated from repositories}

In the future there may exist large general databases with materials
data from which data can be extracted for a particular system.  In
most cases the extracted data will not be model parameters but
original experimental data or data generated from DFT calculations.
Such data can be used for testing the database and also for
assessments and maybe a simple way to define such informations should
be included here.

The experimental data could be described like POP files used in
Thermo-Calc or the ``enter many\_equilibria'' command available in OC.
This means to set the conditions for which the experimental data has
been generated and then recalculate the same information using the
software and available database.  Possibly the model parameters for
the phases could be adjusted to obtain a better fit.

Examples how one could describe
conditions and experimental data for some different kinds of equilibria:
\begin{itemize}
\item An experimental tie-line in a binary two-phase region. After the
  enter command a name is given and the Y means the following commands
  refer to the new equilibrium.
\begin{verbatim}
ENTER EQUILIBRIUM AB1 Y
SET CONDITION T=1273 P=1E5
SET STATUS PHASE *=SUS
SET STATUS PHASE FCC BCC=FIX 0
EXPERIMENT X(FCC,A)=0.2:.01 x(BCC,A)=0.3:0.01
\end{verbatim}

\item The experimental value of the congruent melting of UO$_2$.  The
  condition x(liq,O)-x(C1\_mo2)=0 means both phases should have the
  same mole fraction of O.
\begin{verbatim}
ENTER EQUILIBRIUM CONGRUENT Y
SET CONDITION P=1E5 X(LIQ,O)-X(C1_MO2,O)=0
SET STATUS PHASE *=SUS
SET STATUS PHASE LIQ C1_MO2=FIX 0
EXPERIMENT T=3340:25, X(O)=.66:.01
\end{verbatim}

\item Tables with experimental data like enthalpies of mixing can be
  entered like below for a binary liquid.  The ``enter table'' command
  has first a ``head'' part where the conditions and phase status are
  specfied (all phases suspended by default) in a simplified way (no
  SET needed).  The values that vary in the table are indicated with a
  symbol @j which means the value is in column j.  Note the leftmost
  column is a ``name'' of the equilibrium which is considered as
  column zero!
\begin{verbatim}
ENTER TABLE_OF_EQUIL
ENTERED 1 LIQUID
CONDITION P=1E5 N=1 T=1100 X(B)=@1
REFERENCE_STATE A LIQ * 1E5
REFERENCE_STATE B LIQ * 1E5
EXPERIMENT HM(LIQ)=@2:5%
TABLE_START
LIQH1 0.1 -3700 
LIQH2 0.2 -6000
LIQH3 0.5 -7800 
LIQH4 0.8 -5000 
LIQH5 0.9 -2800 
TABLE_END
\end{verbatim}
As liquid is the only entered phase it is not really necessary to
specify it in the experiment.

\end{itemize}

Extending to these kinds of data we must also define how to specify
various kinds of state variables.

\subsection{Material}\label{sec:material}

In some cases data is characterized for a material which may consist
of several phases.  In order to accommodate such data, particularly for
thermo-physical use, one may use the keyword MATERIAL.  The data for this
keyword is

\begin{verbatim}
MATERIAL <name> <composition> '<comments>' !
\end{verbatim}

The name is 1 to 24 alphanumerical characters or underscore, it must
start with a letter.  The composition is specified in mass percent of
the components, normally the elements but species may be used.  The
chemical symbol of the component is followed by the mass percent, the
major component is specified by giving the value as *.  The comments
may be any text.  

Examples

\begin{verbatim}
MATERIAL PIG_IRON C 4 SI 2.5 MN 0.4 FE * !
MATERIAL LIMESTONE CACO3 3 CAC1O3 * 'Almost pure' !
\end{verbatim}

Such materials definitions may be helpful for specifying the limits of
accuracy for the database.

\begin{thebibliography}{00Zzz}
\bibitem{78Hil} Reasons to model the Gibbs energy.
\bibitem{80Hil} M. Hillert, CALPHAD, {\bf 4} (1980) 1--12
\bibitem{86Hil} M. Hillert, Met Trans A {\bf 17A} (1986) 1878--1879
\bibitem{86Pel} A.D. Pelton and C.W. Bale, Met Trans A {\bf 17A}
  (1986) 1211--1215
\bibitem{91Din} A.T. Dinsdale, CALPHAD, {\bf 15} (1991) 317--425 
\bibitem{TC} J.-O. Anderssson et al. CALPHAD, {\bf 26} (2002) 273--312
\bibitem{FactSage} C.W. Bale et al. CALPHAD, {\bf 26} (2002) 189--228
\bibitem{OC} B. Sundman et al. Intr Mat and Manu Innov {\bf 4:1} (2015)
\bibitem{01Pel} A.D. Pelton, CALPHAD, {\bf 25} (2001) 319--328
\bibitem{07Luk} H.L. Lukas et al. {\em Computational Thermodynamics,
  the CALPHAD method}, Camb. Univ. Press (2007)
\bibitem{13Wei} Wei Xiong, Thesis or magnetic model in Cr-Fe paper 
\end{thebibliography}

\newpage

\section{Appendix A: Example of an interchange}

On the following pages an example of the use of the interchange format
is given in order to describe a database for ??.  

{\em new example needed}

{\small
\begin{verbatim}
to be generated ...
\end{verbatim}
}

\newpage

\section{Appendix B.  Summary of keywords}

\begin{verbatim}
VERSION <date> <lower T limit> <upper T limit>!

ELEMENT <element name> <SER state> <atomic mass in g/mol>
    <enthalpy at 298.15 K - enthalpy at 0 K>
    <entropy  at 298.15 K> !

SPECIES <species name> /<element name> <stoichiometric factor>/ !

TABLE_OF_MODELS /<code> '<explanation>'/!

PHASE <phase name> <model code> /<addition code>/ /<detail code>/ !

CONSTITUENT <phase name> : <species on first sublattice> 
/ : <species on other sublattices in sublattice order>/ : !

TABLE_OF_IDENTIFIERS /<identifier> '<explanation>'/!

PARAMETER <identifier> ( <phase name>, <component array> ; <degree> )
    <expression> <bibligraphic id> !

FUNCTION <name> <expression> !

DATABASE_INFORMATION free text !

BIBLIOGRAPHY /<bibligraphic identifier> '<text>'/ !

SPECIAL <software> <any text and keywords including single !> !!

DEFAULTS /option=value/ !

ASSESSED_SYSTEMS assessed systems with a hyphen between the elements in
     alphabetical order. !

INCLUDE_FILE filename !

CHECKSUM <value> !

\end{verbatim}

\newpage

\section{Appendix C.  Definition of model codes}

All model codes in the version (\today) ~of the PDB format:

\begin{tabular}{l p{130mm}}
Model code & Meaning\\\hline
IDEAL & no sublattices and no excess parameters\\
SUBRKM & Substitutional model with Redlich-Kister Mugianu (RKM) excess model\\
SUBPK  & Substitutional model with polynom model and Kohler extrapolation\\
SUBMIX & Substitutional model with mixed excess model\\
CEFj   & Sublattice model with j sublattices, $1\leq j\leq 9$, RKM excess model\\
CEFjD & as CEFj but also a disordered substitutional model added, RKM excess\\
CEF4F & A 4 sublattice model for ordered fcc or hcp with permutated parameters only once, RKM excess\\
CEF4B & A 4 sublattice model for ordered bcc with permutated parameters only once, RKM excess\\
CEF5F & A 4 sublattice model for ordered fcc or hcp with extra interstitial sublattice
and permutated parameters only once, RKM excess\\
CEF5B & A 4 sublattice model for ordered bcc with extra interstitial sublattice
and permutated parameters only once, RKM excess\\
CEF5FD & A 4 sublattice model for ordered fcc or hcp with extra interstitial sublattice
and permutated parameters only once and a disordered substitutional model for
the ordered sublattices, RKM excess\\
CEF5BD & A 4 sublattice model for ordered bcc with extra interstitial sublattice
and permutated parameters only once and a disordered substitutional model for
the ordered sublattices, RKM excess\\
I2SL & The 2 sublattice ionic liquid model.\\\hline
\end{tabular}

\bigskip

The Gibbs energy expression for these models are found below.  The
symbols used in the Gibbs energy expressions are:

\begin{tabular}{l p{150mm}}
Symbol & Explanation\\\hline
$R$          & gas constant\\
$T$          & absolute temperature\\
$G_M^{\alpha}$ & is the Gibbs energy of $\alpha$ phase for one formula
    unit.  The formula unit is important because phases can have vacancies
    as constituent and thus the total number of moles of atoms may vary.
    In fact the normal model for an ideal gas is per mole formula units, 
    not per mole of atoms.\\
$^{\circ}G_i^{\alpha}$ & is the Gibbs energy of formation of species $i$ in the
$\alpha$ phase from the reference states of the elements.\\
$^{\circ}G_I^{\alpha}$ & is the Gibbs energy of formation of the endmember $I$
    in the $\alpha$ phase from the reference states of the elements.\\
$y_{i}$ or $y_{is}$ & the constituent fraction of $i$ (on sublattice $s$).\\\hline
\end{tabular}

\bigskip

\begin{itemize}
\item model code IDEAL

  The constituents can be any kind of species, including vacancies and
  ions.
  \begin{eqnarray}
    G_M &=& \sum_i y_i~^{\circ}G_i + RT y_i\ln(y_i)\label{eq:ideal}
  \end{eqnarray}

%---------------------------------------------------
\item model code SUBRKM

  The constituents can be any kind of species, including vacancies and
  ions.

  The substitutional model using Redlich-Kister Muggianu extrapolations.
  \begin{eqnarray}
    G_M &=& \sum_i y_i~^{\circ}G_i + RT y_i\ln(y_i) + ~^EG_M\\
    ^EG_M &=& \sum_i \sum_{j>i} y_i y_j (L_{ij} + \sum_k y_k (L_{ijk} + \cdots))
  \end{eqnarray}

  The binary interaction parameters are described with Redlich-Kister polynom.
  \begin{eqnarray}
    L_{ij} &=& \sum_{\nu} (y_i-y_j)^{\nu} ~^{\nu}L_{ij}\label{eq:rk2}
  \end{eqnarray}
  The ternary parameters can be composition dependent according to Hillert's
  proposal\cite{80Hil}:
  \begin{eqnarray}
    L_{ijk}&=&\sum_{\nu}z_i ~^iL_{ijk}+z_j ~^jL_{ijk}+z_k ~^kL_{ijk}\label{eq:tern}\\
    z_i &=& y_i +(1-y_i-y_j-y_k)/3\nonumber\\
    z_j &=& y_j +(1-y_i-y_j-y_k)/3\\
    z_k &=& y_k +(1-y_i-y_j-y_k)/3\nonumber
  \end{eqnarray}
  The use of $z_i$ instead of $y_i$ makes this formulation symmetric
  also in multicomponent systems.

%---------------------------------------------------
\item model code SUBPK

  The constituents can be any kind of species, including vacancies and
  ions.

  The substitutional model polymer for binary excess with Kohler
  extrapolations.
  \begin{eqnarray}
    G_M &=& \sum_i y_i~^{\circ}G_i + RT y_i\ln(y_i) + ~^EG_M\\
    ^EG_M &=& \sum_i \sum_{j>i} z_i z_j (L_{ij} + \sum_k z_k (L_{ijk} + \cdots))
  \end{eqnarray}
  where $z_i, z_j$ are extrapolated from ternary to ``binary''
  compositions using a Kohler model.  
  \begin{eqnarray}
    L_{ij} &=& ??\label{eq:polex}
  \end{eqnarray}

  {\em I am not sure how to specify the independent binary variable
    and what fractions should be used for any ternary term}

%---------------------------------------------------
\item model code SUBMIX

  The constituents can be any kind of species, including vacancies and
  ions.

  The substitutional model using mixed ternary extrapolations (Toop,
  Koohler, Redlich-Kister etc).  Using the elegant method proposed by
  Pelton\cite{01Pel} one can define for each ternary subsystem which
  extrapolation method should be used.  However this means one must
  for each ternary combination of constituents in a phase give which
  extrapolation model that should be used.  I am not sure how this can
  be done most simply.

  \begin{eqnarray}
    G_M &=& \sum_i y_i~^{\circ}G_i + RT y_i\ln(y_i) + ~^EG_M\\
    ^EG_M &=&\sum_i \sum_{j>i} z_i z_j (L_{ij} + \sum_k z_k (L_{ijk} + \cdots))
  \end{eqnarray}
  where $z_i, z_j$ are extrapolated from ternary to ``binary''
  compositions using a Kohler model.
  
  {\em I am not sure what fraction should be used for the ternary term
  
  For a quaternary system A-B-C-D where A is Toop element in the A-B-C
  system and B a Toop element in B-C-D and for the rest one use Kohler
  extrapolations I could imagine this specified as:
  
  PHASE BCC SUBMIX(TK(A B C) TK(B C D) KK(*)) !
  
  where TK(A B C) means A is the Toop element and for B and C use
  Kohler.  KK(*) means use Kohler model for all other ternaries.  But
  I am not sure if this is the sufficient or the simplest way.}

  {\em I am not sure if anyone used Toop or Kohler combined with
    sublattices and how one should handle that.}
  
%---------------------------------------------------
\item model code I2SL (Partially Ionic 2 sublattice Liquid model)

  The constituents must be cations on the first sublattice and there
  must not be any cations on the second sublattice.

  This is a 2 sublattice model with cations on the first sublattice
  and anions, vacancies and neutrals on the second sublattice.  The
  site ratios, $P$ and $Q$, depend on the average charge on the
  opposite sublattice.

  \begin{eqnarray}
    G_M &=& \sum_i \sum_j y_iy_j~^{\circ}G_{ij} + Q(y_{\rm Va}\sum_i y_i^{\circ}G_i + 
    \sum_k y_k ^{\circ}G_k) + RT P\sum_i y_i\ln(y_i)+\nonumber\\&&
    RTQ(\sum_j y_j\ln(y_j)+y_{\rm Va}\ln(y_{\rm Va}) + \sum_k y_k\ln(y_k))+ ~^EG_M\\
    P &=& \sum_j \nu_jy_j + Qy_{\rm Va}\\
    Q &=& \sum_i \nu_iy_i\\
    ^EG_M &=& \sum_{i_1}\sum_{i_2}\sum_j y_{i_1}y_{i_2}y_j L_{i_1,i_2:j}+\sum_i\sum_{j_1}\sum_{j_2}y_iy_{j_1}y_{j_2}L_{i:j_1,j_2}+\nonumber\\&&
    \sum_i\sum_jy_iy_jy_{Va}L_{i:j,Va}+\sum_i\sum_j\sum_ky_iy_jy_kL_{i:j,k}+\nonumber\\&&
    Q\sum_{i_1}\sum_{i_2} y_{i_1}y_{i_2}y_{Va}^2 L_{i,i_2:Va}+Q\sum_i\sum_ky_iy_ky_{Va}L_{i:k,Va}+\nonumber\\&&
    Q\sum_{k_1}\sum_{k_2}y_{k_1}y_{k_2}L_{k_1,k_2}\label{eq:i2slx}
  \end{eqnarray}
  where $i$ denote a cation with charge $+\nu_i$, $j$ an anion with
  charge $-\nu_j$, Va vacancies and $k$ neutrals.
  
  For a discussion of the excess term and other details see
  Lukas~\cite{07Luk}.  Note eq.~\ref{eq:i2slx} is wrong in
  Lukas~\cite{07Luk} as the factors Q have are missing.

  {\em Eq.~\ref{eq:i2slx} has been corrceted.}

%---------------------------------------------------
\item model code KFGL

  The constituents are ``cells'' like A-O-A and A-O-B and a special
  ordering of the constituents is needed.

  The cell model proposed by Kaphor-Frohberg and extended by Guy and Lehmann
  \begin{eqnarray}
    G_M &=& {\rm complicated}
  \end{eqnarray}

%---------------------------------------------------
\item model code GUTS

  The constituents can be any kind of species, including vacancies and
  ions??

  Used by FactSage
  \begin{eqnarray}
    G_M &=& {\rm I~do~not~know}
  \end{eqnarray}

%---------------------------------------------------
\item model code CEFj

  The constituents can be any kind of species, including vacancies and
  ions.

  This is the normal Compound Energy Formalism (CEF) and the number of
  sublattices is given as j, $1 \leq j \leq 9$.  The value j=1 can be
  used when the formula unit should have more than one site.  The site
  ratios are specified within parenthesis like
  
  PHASE SIGMA CEF5(2.0 4.0 8.0 8.0 8.0) !
  
  \begin{eqnarray}
    G_M &=& \sum_I \Pi(y_{i\in I}) ~^{\circ}G_I + RT \sum_s a_s \sum_i y_{is}\ln(y_{is}) + ~^EG_M
  \end{eqnarray}
  where $\Pi(y_{i\in I})$ is the product of the constituent fractions
  of the endmember $I$, $~^{\circ}G_I$ is the Gibbs energy of
  formation of the endmember (compound) $I$ from the reference state
  of the elements.  $a_s$ are the site ratio for sublattice $s$.

  For detals especially about the excess model see Lukas~\cite{07Luk}.

  {\em As far as I know there has never been any assessment using CEF with
  any other excess models than Redlich-Kister\_Muggianu.}

%---------------------------------------------------
\item model code CEFjD

  The constituents can be any kind of species, including vacancies and
  ions.

  This adds a disordered substitutional part to a phase modeled with j
  sublattices, $1\leq j \leq 9$, for example $\sigma$ or Laves\_C14.
  The configurational entropy of the substitutional part is
  ignored.
  \begin{eqnarray}
    G_M &=& G_M^{\rm ord}(y) + G_M^{\rm dis}(x) - RT\sum_i x_i\ln(x_i)
  \end{eqnarray}

  Example:

  PHASE SIGMA CEF5D(2.0 4.0 8.0 8.0 8.0) !
  
%---------------------------------------------------
\item model code CEF4F and CEF4B 

  The constituents can be any kind of species, including vacancies and
  ions.  The set of constituents must be identical on all sublattices.

  This is a special case of CEF describing ordering on FCC, HCP or BCC
  using 4 sublattices (tetrahedron).  The number of sites on all 4
  sublattices are equal.  There can be several ordered forms with
  different fractions of constituents on the sublattices.  There is
  also a totally disordered state with same fractions on all
  sublattices.  The latter is equivalent to a normal disorderd FCC,
  HCP or BCC.

  PHASE FCC\_4SL CEF4F(0.25 0.25 0.25 0..25) !

  The structure imposes several restrictions on the model parameters.
  \begin{eqnarray}
    G_M &=& \sum_I \Pi(y_{i\in I}) ~^{\circ}G_I + RT \sum_s a_s \sum_i y_{is}\ln(y_{is}) + ~^EG_M
  \end{eqnarray}
  The constituents can be any kind of species, including vacancies and
  ions.

  For model codes CEF4F, CEF4B, CEF5F and CEF5B the model parameters
  have several permutations, for example
  \begin{eqnarray}
    ^{\circ}G_{\rm A:A:A:B}=^{\circ}G_{\rm A:A:B:A}=^{\circ}G_{\rm A:B:A:A}=^{\circ}G_{\rm B:A:A:A}
  \end{eqnarray}
  and such parameters need only be given once in the database, the
  software will take care of the permutations.

  NOTE permutations applies also to interaction parameters!  So it can
  be quite complicated to implement.


%---------------------------------------------------
\item model code CEF5F and CEF5B 

  The constituents can be any kind of species, including vacancies and
  ions.  The set of constituents must be identical on the first 4
  sublattices.

  This is the same as CEF4F and CEF4B but has an extra interstitial
  sublattice.

  PHASE BCC\_4SL CEF5B(0.25 0.25 0.25 0.25 3) ! 

  \begin{eqnarray}
    G_M &=& \sum_I \Pi(y_{i\in I}) ~^{\circ}G_I + RT \sum_s a_s\sum_i y_{is}\ln(y_{is}) + ~^EG_M
  \end{eqnarray}

%---------------------------------------------------
\item model code CEF4FD, CEF4BD, CEF5FD and CEF5BD

  The constituents can be any kind of species, including vacancies and
  ions.  The set of constituents must be identical on the first 4
  sublattices.

  The final D means that the model parameters describing the
  disordered state has been extracted to a single sublattice model.
  The parameters for the ordered 4 sublattice model describe only the
  contribution due to ordering.
  \begin{eqnarray}
    G_M &=& G_M^{\rm dis}(x) + \Delta G_M^{\rm ord}(y)\\
    \Delta G_M^{\rm ord}(y) &=& G_M^{\rm 4sl}(y) - G_M^{\rm 4sl}(y=x)
  \end{eqnarray}
  where the mole fractions $x$ are calculated from the constitution,
  $y$.  The last equation above means that the Gibbs energy for the
  ordered part is calculated twice, once with the original
  constitution, once with the constitution equal to the disordered
  state.  When the phase is disordered the contribution from $\Delta
  G_M^{ord}$ is thus zero.

  {\em In OC I use a D after the normal parameter identfier to specify
    that the parameter belong to the disordered set, for example
    TCD(FCC,FE) is the magnetic ordering $T$ for disordered FCC-Fe.
    Parameters for the disordered set could also be detected by the
    number of sublattices in the constituent array of the parameter
    but I think it is useful to have a final D alsor for the
    disordered parameters as that may avoid a lot of errors.}

%---------------------------------------------------
\item model code BONDj

  The constituents can be any kind of species, including vacancies and
  ions.

  This is a special variant of CEF with j sublattices, $2\leq j \leq
  9$, using bond energies between constituents on pairs of
  sublattices.

  PHASE FCC BOND4(1 1 1 1) !

  \begin{eqnarray}
    G_M &=& \sum_i x_i ~^{\circ}G_i + \sum_{s=1}^n \sum_{r=2}^{n-1} (n-1) y_{is} y_{jr}G_{ij}+RT\sum_s \sum_i y_{is}\ln(y_is)
  \end{eqnarray}
  where $~^{\circ}G_{ij}$ is a Gibbs energy for the bond between
  constituent $i$ in sublattice $s$ and constituent $j$ in sublattice
  $r$.  Such a parameter will have a wildcard, ``*'', for the
  constituent in all other sublattices, like G(BOND,*:FE:NI:*) to give
  the bond energy between Fe in sublattice 2 and Ni in sublattice 3.
\end{itemize}

\newpage

\section{Appendix D.  Definition of addition codes}

These define models of additions which at present mainly is for the
ferromagnetic transition but may be expanded to include many new
things like heat capacity models at low T, volumes, strain-stress
relations etc.  All of these may include parameters with specific
identifiers that depend on $T, P$ and constitution of the phase.

\begin{itemize}
\item Addition code IMAGB and IMAGF

  Inden magnetic model for BCC and FCC.  The expression for this can
  be found in almost any Calphad paper or in Lukas\cite{07Luk}.

  The value of the magnetic contribution will depend on parameters
  with identifiers TC and BMAG.

\item Addition code IWMAGB and IWMAGF

  Inden magnetic model as modified by Qing-Wei with separate Curie and
  Neel temperatures and individual Bohr magneton number.\cite{13Wei}

  The value of the magnetic contribution will depend on parameters
  with identifiers CTA, NTA and IBM\&A.

\item Addition DEBYE and EINSTEIN

  Whatever is proposed for extraplating $C_P$ and $G$ down to 0~K.

  The value of the Debye/Einstein $T$ contribution will depend on parameters
  with identifier THET

\item Addition MURN

  The Murnaghan model for pressure dependence.

  There are no identifiers for this model as yet.

\end{itemize}

\newpage

\section{Appendix E.  Definition of detail codes}

These are for things related to the phase but maybe more informative
or software specific and can possibly be ignored.

\begin{itemize}
\item Detail code STRUCT

After this and within parenthesis one can give free text like:
\begin{verbatim}
PHASE HALITE CEF2(1.0 1.0) IMAGF DETAIL(STRUCT(B2, PERICLASE)) !
\end{verbatim}


\item Detail code MGAP 

  specifies there is a miscibility gap in a phase.  It also gives the
  major constituents in all sublattices and possibly a pre and suffix
  to the phase name.
\begin{verbatim}
DETAIL(prefix : /major constituent in sublattice :/ suffix)
\end{verbatim}

  PHASE FCC\_A1 CEF2( 1.0 1.0) IMAGF MGAP(MC :TI V:C N : CARB) !

  Here the phase has a CEF model with 2 sublattices.  It has the Inden
  magnetic addition and there is a detail that there can be a
  miscibility gap with major constituents Ti and V in the first
  sublattice and C and N in the second.  The text CC (meaning Cubic
  Carbide) is the prefix and CARB is a suffix that could be added to
  the phase name for this composition set.  If no pre- or suffix
  needed just start with a : and give nothing after the last :.

  Note that a phase can have miscibility gaps even if there is no MGAP
  keyword!  Of corse miscibility gaps are conveniently detected by
  global minimization but sometimes one may want to know which
  composition set has a specific major constitution.

  There are also cases when a calculation cannot rely on global
  minimization, for example if there is no condition on $T$.

\item Detail code DCON

  This can be used to specify default constituents (used for the first
  composition set).  For example

  PHASE FCC\_A1 CEF2( 1.0 1.0) MAGIF MGAP(CC :Ti V:C N:) DCON(FE NI:VA) !

  This may be used instead of the \% used for major constituents in
  the current TDB format, see section~\ref{sec:phaserange}.

\end{itemize}

\newpage

\section{Appendix F.  Parameter identifiers}

A parameter identifier should not be longer than 4 letters (excluding
possible element or constituent specification).  In the table below a
parameter identifier may depend on $T$ or $P$ and that is indicated by
a *.  If is is specific for an element that is marked by an \&X and if
specific for a constituent \&X\#s.

\begin{tabular}{lccccl}
Identifier & Spec & Unit & T & P & Description\\\hline
G          & -    & J/FU & * & * &The Gibbs energy per formula unit\\
L          & -    & J/FU & * & * &Identical to G but mainly used for interactions\\ 
TC         & -    & K    & - & * & The Critical $T$ for magnetic ordering\\
BMAG       & -    & B.mag& - & * & The average Bohr magneton number\\
CTA        & -    & K    & - & * & The Curie $T$ for ferro-magnetic ordering\\
NTA        & -    & K    & - & * & The Neel $T$ for anti-ferromagnetic ordering\\
%IBM        & \&X\#s &B.mag&- & * & The Bohr magneton for constituent A in sublattice s\\
IBM        & \&A\#s &B.mag&- & * & The Bohr magneton for constituent A in sublattice s\\
THET       & - & K & - & * & Debye or Einstein $T$\\
MQ         & \&X  & ? & * & * & LN mobility of X \\
RHO        & - & ? & * & * & Electric resistivity\\
MAGS  & - & ? & * & * & Magnetic susceptibility\\
GTT   & - & K & - & * & Glass transition temp\\
VISC  & - & ? & * & * & Viscosity\\
LPX   & - & m & * & * & Lattice parameter in X direction\\
LPY   & - & m & * & * & Lattice parameter in Y direction\\
LPZ   & - & m & * & * & Lattice parameter in Z direction\\
LAA   & - &degree& * & * & Lattice angle $\alpha$\\
LAB   & - &degree& * & * & Lattice angle $\beta$\\
LAG   & - &degree& * & * & Lattice angle $\gamma$\\
EC11  & - & ? &* & * & Elastic constant C11\\
EC12  & - & ? &* & * & Elastic constant C12\\
EC44  & - & ? &* & * & Elastic constant C44\\
V0    & - & m3 &- & - & Molar volume at 298.15~K and 1 bar\\
VA    & - & m/K & * & - & Thermal expansion at 1 bar\\
VB    & - & m2/N & * & * &Bulk modulus\\\hline
\end{tabular}

\bigskip

A parameter can have a suffix D if it belongs to a disordered part.
Such a parameter will also have fewer sublattices in the constituent
array.

The parameter identifiers are used to describe how a property varies
with composition and possibly $T$ and $P$.  Most properties are for
the whole phase but some, like the mobility, describe how an element
or constituent specific property varies with $T, P$ and composition of
the phase.

Many parameters may not influence the Gibbs energy but are used for
simulations and it is convenient to store them together with the
thermodynamic data, as long as they are phase specific.  Some
properties may overlap, like the volume and lattice parameter.  But
hopefully things like this will be sorted out eventually.

An element specific property like ln(mobility) of Fe in a phase is
given with the identifier MQ\&FE and this can vary with the
composition and $T$ and $P$.  For example ln(mobility) of Fe in the
BCC phase in the Cr-Fe system can have the following parameters:

\begin{tabular}{ll}
MQ\&FE(BCC,FE) & ln(mobility) of Fe in pure BCC-Fe\\
MQ\&FE(BCC,CR) & ln(mobility) of an Fe atom in pure BCC-Cr\\
MQ\&FE(BCC,CR,FE) & describe composition dependence\\
\end{tabular}

These parameters can depend on $T$ and $P$.  The mobility is
calculated as:
{\small
\begin{eqnarray}
MQ\&Fe(BCC) = x_{Fe}MQ\&Fe(BCC,FE)+x_{Cr}MQ\&Fe(BCC,CR)+x_{Fe}x_{Cr}MQ\&Fe(BCC,CR,FE)
\end{eqnarray}}
where MQ\&FE(BCC) is the value of ln(mobility) of FE in BCC for the
current values of $T, P$ and composition.

The Bohr magneton number of Fe$^{+3}$ in sublattice 3 of magnetite as
a function of composition is described by the constituent specific
property IBM\&Fe+3\#3.

{\em We are creating a database with a large number properties where we
need model parameters when the phase is not stable!}

\newpage

\section{Appendix G.  The constituent array for interaction parameters}

{\em I have not thought much about these}

Sign depend on alphabetical order of constituents for odd
Redlich-Kister terms

Toop model

Polynomial models for constituents ...

\newpage

\section{Appendix H, The function of $T$ and $P$}\label{sec:exptype}

{\em I think there is no need for an EXPTYPE keyword as we probably do
  not need any other type than the old 5.  Dealing with solution
  phases it seems not very meaningful to have C$_P$ expressions and
  with fixed $T$ coefficients they can always be integrated to a Gibbs
  energy.

  The rather lengthy description below is due to the fact that in 1990
  all software except Thermo-Calc stored the coefficients for these
  expressions in tables, usually with a fixed number of coefficients.}

In the TDB files the keyword EXPTYPE could be used to specify the way
functions of $T$ and $P$ were entered.  But only exptype 5 are used in
current TDB files so this keyword is no longer needed.

As some software calculate first and second derivatives of $T$ and $P$
using these functions there should be some restrictions on the
complexity of the expressions that are allowed.  The proposed syntax
is that of the old exptype 5:

\begin{verbatim}
<low temperature limit> / <function> ; <high temperature limit> <Y or N> /
\end{verbatim}

The syntax of the function part is given below.  A semicolon must be
use to terminate the function and it is followed by a space and the
high temperature limit.  Finally a ``Y'' or an ``N'' must be given.  N
means that there are no more temperature ranges and Y that there will
be another function above the current high temperature limit.  The
information between the slashes is repeated until it is terminated by
an N.

{\em In the TDB format a `` ,'' (space followed by a comma) was
  allowed for the low and hight T limit.  The , meant that the default
  value was taken, for the low limit 298.15 and for the high limit
  6000~K.  There is no reason not to allow this also in the new format
  as it simplifies a little for the manager and the default low and
  high limits can be set with the VERSION keyword.}

The function is written similar to a Fortran like statement but
without the use of parenthesis for grouping terms.  The basic entity of
the function is called a ``simple term''.  A simple term is:

\begin{verbatim}
<real number> * <symbol name> **<power> *T** <power> *P** <power>
\end{verbatim}

The text between $<$ and $>$ describe the item which should appear
there.  The other items must be given literally and have their usual
meaning i.e.  * is multiplication and ** is exponentiation, T is the
temperature and P is the pressure.  A symbol can be void or another
expression, see the FUNCTION keyword below.

If the simple term does not depend on any symbol or T or P that part
can be omitted.  The power can only be an integer but non-integer powers
can be handled with a LOG and EXP pair as described below.  Negative
powers must be surrounded by parenthesis.  Redundant parts of a simple
term can be omitted and if the power is unity the exponentiation can
be omitted altogether.  Note that the real number must be the first
item and any symbol must precede the T and P.  Examples of simple terms
are:

\begin{verbatim}
1.15*T      -V1     1E-12*P**2    -456754.65*T**(-1)*P        10*R*T
\end{verbatim}

In order to include the logarithm and exponential in these functions
it is allowed to multiply a simple term with the logarithm or       
exponential of another simple term.  This more generalized entity is 
called a term and it is defined

\begin{verbatim}
<term> = <simple term> or
	 <simple term> * <symbol> or
	 <simple term> * LN( <simple term> ) or
	 <simple term> * LOG( <simple term> ) or
	 <simple term> * EXP( <simple term> ) or
	 <simple term> * ERF( <simple term> )
\end{verbatim}

A term can be equal to one of the lines above.  It is illegal to have
both exponentials and logarithms in the same term.  It would not be
difficult to add mode unary functions.

Examples of terms are:

\begin{verbatim}
+1.15*T*LN(T)       +1E-6*LOG(-32000*T**(-1))    -5*V3*T*EXP(V4*P)
\end{verbatim}

A function of this type is thus a number of terms written after
each other in order to form an expression.  All terms except the first
must be preceded by a sign.  Examples of expressions are:

\begin{verbatim}
-10000 +10*T+1.15*T*LN(T) +7.5E-5*P +134567*T**(-1) -1.13E-12*T*P

F1+2.5*R*T*LN(T)+R*T*LN(P)
\end{verbatim}
where F1 is another FUNCTION.  Note that there must be a semicolon to
separate the function from the following high temperature limit.  Note
also that the symbols can denote a numeric value or another function.
See the FUNCTION keyword.

Some of the restrictions on a function are:                 
\begin{itemize}
\item The order of the factors in a simple term must be followed,
\item no parenthesis allowed except for exponentials, logarithms or
  negative powers,
\item no division allowed,                                             
\item no spaces allowed between a sign and a numeric value,            
\item only one symbol in each simple term,
\item only integers as powers.
\end{itemize}

Some of these restrictions are due to simplify the parsing of the
functions and some are due to the requirement that it must be possible
to calculate quickly the value of the function as well as its first
and second derivatives with respect to $T$ and $P$.

\end{document}

